\appendix
\chapter{Source code} \label{ch:source}

This appendix contains the source of the controller as given by revision no. 6 to the SVN repository
\begin{verbatim}//repos.gbar.dtu.dk/mhha/Controller/\end{verbatim}
There are two files, risoe\_controller.f90 and risoe\_controller\_fcns.f90 in this Fortran90 code. This type2 DLL for HAWC2 has been compiled with Intel Fortran 13.0.

\section{Main routines in risoe\_controller.f90}
{\scriptsize
\begin{verbatim}
subroutine init_regulation(array1,array2)
use risoe_controller_fcns
implicit none
!DEC$ ATTRIBUTES DLLEXPORT, C, ALIAS:'init_regulation'::init_regulation
real*8 array1(1000),array2(1)
! Local vars
integer*4 i,ifejl
character text32*32
real*8 minimum_pitch_angle
logical findes
! Input array1 must contain
!
!       ; Overall parameters
!    1: constant   1   ; Rated power [kW]
!    2: constant   2   ; Minimum rotor speed [rad/s]
!    3: constant   3   ; Rated rotor speed [rad/s]
!    4: constant   4   ; Maximum allowable generator torque [Nm]
!    5: constant   5   ; Minimum pitch angle, theta_min [deg],
!                      ; if |theta_min|>90, then a table of <wsp,theta_min> is read
!                      ; from a file named 'wptable.n', where n=int(theta_min)
!    6: constant   6   ; Maximum pitch angle [deg]
!    7: constant   7   ; Maximum pitch velocity operation [deg/s]
!    8: constant   8   ; Frequency of generator speed filter [Hz]
!    9: constant   9   ; Damping ratio of speed filter [-]
!   10: constant  10   ; Frequency of free-free DT torsion mode [Hz], if zero no notch filter used
!       ; Partial load control parameters
!   11: constant  11   ; Optimal Cp tracking K factor [Nm/(rad/s)^2],
!                      ; Qg=K*Omega^2, K=eta*0.5*rho*A*Cp_opt*R^3/lambda_opt^3
!   12: constant  12   ; Proportional gain of torque controller [Nm/(rad/s)]
!   13: constant  13   ; Integral gain of torque controller [Nm/rad]
!   14: constant  14   ; Differential gain of torque controller [Nm/(rad/s^2)]
!       ; Full load control parameters
!   15: constant  15   ; Generator control switch [1=constant power, 2=constant torque]
!   16: constant  16   ; Proportional gain of pitch controller [rad/(rad/s)]
!   17: constant  17   ; Integral gain of pitch controller [rad/rad]
!   18: constant  18   ; Differential gain of pitch controller [rad/(rad/s^2)]
!   19: constant  19   ; Proportional power error gain [rad/W]
!   20: constant  20   ; Integral power error gain [rad/(Ws)]
!   21: constant  21   ; Coefficient of linear term in aerodynamic gain scheduling, KK1 [deg]
!   22: constant  22   ; Coefficient of quadratic term in aerodynamic gain scheduling, KK2 [deg^2]
!                      ; (if zero, KK1 = pitch angle at double gain)
!   23: constant  23   ; Relative speed for double nonlinear gain [-]
!       ; Cut-in simulation parameters
!   24: constant  24   ; Cut-in time [s], if zero no cut-in simulated
!   25: constant  25   ; Time delay for soft start [1/1P]
!       ; Cut-out simulation parameters
!   26: constant  26   ; Cut-out time [s], if zero no cut-out simulated
!   27: constant  27   ; Time constant for 1st order filter lag of torque cut-out [s]
!   28: constant  28   ; Stop type [1=linear two pitch speed stop, 2=exponential pitch speed stop]
!   29: constant  29   ; Time delay for pitch stop 1 [s]
!   30: constant  30   ; Maximum pitch velocity during stop 1 [deg/s]
!   31: constant  31   ; Time delay for pitch stop 2 [s]
!   32: constant  32   ; Maximum pitch velocity during stop 2 [deg/s]
!       ; Expert parameters (keep default values unless otherwise given)
!   33  constant  33   ; Lower angle above lowest minimum pitch angle for switch [deg]
!   34: constant  34   ; Upper angle above lowest minimum pitch angle for switch [deg]
!   35: constant  35   ; Ratio between filtered and reference speed for fully open torque limits [%]
!   36: constant  36   ; Time constant of 1st order filter on wind speed used for minimum pitch [1/1P]
!   37: constant  37   ; Time constant of 1st order filter on pitch angle for gain scheduling [1/1P]
!  	38: constant  38   ; Proportional gain of DT damper [Nm/(rad/s)], requires frequency in input 10
!
! Output array2 contains nothing for init
!
! Overall parameters
Pe_rated                 =array1( 1)*1.d3
omega_ref_min            =array1( 2)
omega_ref_max            =array1( 3)
max_lss_torque           =array1( 4)
minimum_pitch_angle      =array1( 5)*degrad
pitch_stopang            =array1( 6)*degrad
PID_pit_var.velmax       =array1( 7)*degrad
omega2ordervar.f0        =array1( 8)
omega2ordervar.zeta       =array1( 9)
DT_mode_filt.f0          =array1(10)
! Partial load control parameters
Kopt                     =array1(11)
if (Kopt*omega_ref_max**2.ge.Pe_rated/omega_ref_max) Kopt=Pe_rated/omega_ref_max**3
PID_gen_var.Kpro         =array1(12)
PID_gen_var.Kint         =array1(13)
PID_gen_var.Kdif         =array1(14)
! Full load control parameters
const_power         =(int(array1(15)).eq.1)
PID_pit_var.kpro(1)      =array1(16)
PID_pit_var.kint(1)      =array1(17)
PID_pit_var.kdif(1)      =array1(18)
PID_pit_var.kpro(2)      =array1(19)
PID_pit_var.kint(2)      =array1(20)
PID_pit_var.kdif(2)      =0.d0
kk1                      =array1(21)*degrad
kk2                      =array1(22)*degrad*degrad
rel_limit                =array1(23)
! Cut-in simulation parameters
t_cutin                  =array1(24)
t_cutin_delay            =array1(25)*2.d0*pi/omega_ref_max
! Cut-out simulation parameters
t_cutout                 =array1(26)
torquefirstordervar.tau  =array1(27)
pitch_stoptype       =int(array1(28))
pitch_stopdelay          =array1(29)
pitch_stopvelmax         =array1(30)*degrad
pitch_stopdelay2         =array1(31)
pitch_stopvelmax2        =array1(32)*degrad
! Expert parameters (keep default values unless otherwise given)
switch1_pitang_lower     =array1(33)*degrad
switch1_pitang_upper     =array1(34)*degrad
rel_sp_open_Qg           =array1(35)*1.d-2
wspfirstordervar.tau     =array1(36)*2.d0*pi/omega_ref_max
pitchfirstordervar.tau   =array1(37)*2.d0*pi/omega_ref_max
! Drivetrain damper
DT_damp_gain             =array1(38)
DT_damper_filt.f0        =DT_mode_filt.f0
pwr_DT_mode_filt.f0      =DT_mode_filt.f0
! Default and derived parameters
PID_gen_var.velmax=0.d0 !No limit to generator torque change rate
Qg_rated=Pe_rated/omega_ref_max
switchfirstordervar.tau=2.d0*pi/omega_ref_max
cutinfirstordervar.tau=2.d0*pi/omega_ref_max
! Wind speed table
if (dabs(minimum_pitch_angle).lt.90.d0*degrad) then
  Opdatavar.lines=2
  Opdatavar.wpdata(1,1)=0.d0
  Opdatavar.wpdata(2,1)=99.d0
  Opdatavar.wpdata(1,2)=minimum_pitch_angle
  Opdatavar.wpdata(2,2)=minimum_pitch_angle
else
  write(text32,'(i)') int(minimum_pitch_angle*raddeg)
  inquire(file='wpdata.'//trim(adjustl(text32)),exist=findes)
  if (findes) then
    open(88,file='wpdata.'//trim(adjustl(text32)))
    read(88,*,iostat=ifejl) Opdatavar.lines
    if (ifejl.eq.0) then
      do i=1,Opdatavar.lines
        read(88,*,iostat=ifejl) Opdatavar.wpdata(i,1),Opdatavar.wpdata(i,2)
        if (ifejl.ne.0) then
          write(6,*) ' *** ERROR *** Could not read lines in minimum '&
                   //'pitch table in file wpdata.'//trim(adjustl(text32))
          stop
        endif
        Opdatavar.wpdata(i,2)=Opdatavar.wpdata(i,2)*degrad
      enddo
    else
      write(6,*) ' *** ERROR *** Could not read number of lines '&
               //'in minimum pitch table in file wpdata.'//trim(adjustl(text32))
      stop
    endif
    close(88)
  else
    write(6,*) ' *** ERROR *** File ''wpdata.'//trim(adjustl(text32))&
             //''' does not exist in the working directory'
    stop
  endif
endif
! Initiate the dynamic variables
stepno=0
time_old=0.d0
! No output
array2=0.d0
return
end subroutine init_regulation
!**************************************************************************************************
subroutine update_regulation(array1,array2)
use risoe_controller_fcns
implicit none
!DEC$ ATTRIBUTES DLLEXPORT, C, ALIAS:'update_regulation'::update_regulation
real*8 array1(1000),array2(100)
! Input array1 must contain
!
!    1: general time                           ; [s]
!    2: constraint bearing1 shaft_rot 1 only 2 ; [rad/s] Generator LSS speed
!    3: constraint bearing2 pitch1 1 only 1    ; [rad]
!    4: constraint bearing2 pitch2 1 only 1    ; [rad]
!    5: constraint bearing2 pitch3 1 only 1    ; [rad]
!  6-8: wind free_wind 1 0.0 0.0 hub height    ; [m/s] global coords at hub height
!
! Output array2 contains
!
!    1: Generator torque reference            [Nm]
!    2: Pitch angle reference of blade 1      [rad]
!    3: Pitch angle reference of blade 2      [rad]
!    4: Pitch angle reference of blade 3      [rad]
!    5: Power reference                       [W]
!    6: Filtered wind speed                   [m/s]
!    7: Filtered rotor speed                  [rad/s]
!    8: Filtered rotor speed error for torque [rad/s]
!    9: Bandpass filtered rotor speed         [rad/s]
!   10: Proportional term of torque contr.    [Nm]
!   11: Integral term of torque controller    [Nm]
!   12: Minimum limit of torque               [Nm]
!   13: Maximum limit of torque               [Nm]
!   14: Torque limit switch based on pitch    [-]
!   15: Filtered rotor speed error for pitch  [rad/s]
!   16: Power error for pitch                 [W]
!   17: Proportional term of pitch controller [rad]
!   18: Integral term of pitch controller     [rad]
!   19: Minimum limit of pitch                [rad]
!   20: Maximum limit of pitch                [rad]
!   21: Torque reference from DT damper       [Nm]
!
! Local variables
real*8 time,omega,omegafilt,domega_dt_filt,wsp,WSPfilt
real*8 omega_err_filt_pitch,omega_err_filt_speed,omega_dtfilt
real*8 ommin1,ommin2,ommax1,ommax2
real*8 pitang(3),meanpitang,meanpitangfilt,theta_min,e_pitch(2)
real*8 kgain_pitch(3,2),kgain_torque(3),aero_gain,x,dummy,y(2)
real*8 Qg_min_partial,Qg_max_partial,Qg_min_full,Qg_max_full
real*8 Qgen_ref,theta_col_ref,thetaref(3),Pe_ref,Qdamp_ref
!**************************************************************************************************
! Increment time step (may actually not be necessary in type2 DLLs)
!**************************************************************************************************
time=array1(1)
if (time.gt.time_old) then
  deltat=time-time_old
  time_old=time
  stepno=stepno+1
endif
!**************************************************************************************************
! Inputs and their filtering
!**************************************************************************************************
omega=array1(2)
! Mean pitch angle
pitang(1)=array1(3)
pitang(2)=array1(4)
pitang(3)=array1(5)
meanpitang=(pitang(1)+pitang(2)+pitang(3))/3.d0
! Wind speed as horizontal vector sum
wsp=dsqrt(array1(6)**2+array1(7)**2)
! Low-pass filtering of the rotor speed
y=lowpass2orderfilt(deltat,stepno,omega2ordervar,omega)
omegafilt=y(1)
domega_dt_filt=y(2)
! Mean pitch angle
! Low-pass filtering of the mean pitch angle for gain scheduling
meanpitangfilt=min(lowpass1orderfilt(deltat,stepno,pitchfirstordervar,meanpitang),30.d0*degrad)
! Low-pass filtering of the nacelle wind speed
WSPfilt=lowpass1orderfilt(deltat,stepno,wspfirstordervar,wsp)
! Minimum pitch angle may vary with filtered wind speed
theta_min=GetOptiPitch(WSPfilt)
switch1_pitang_lower=switch1_pitang_lower+theta_min
switch1_pitang_upper=switch1_pitang_upper+theta_min
!**************************************************************************************************
! Speed ref. changes max. <-> min. for torque controller and remains at rated for pitch controller
!**************************************************************************************************
if (omegafilt.gt.0.5d0*(omega_ref_max+omega_ref_min)) then
  omega_err_filt_speed=omegafilt-omega_ref_max
else
  omega_err_filt_speed=omegafilt-omega_ref_min
endif
!**************************************************************************************************
! PID regulation of generator torque
!**************************************************************************************************
! Limits for full load
if (const_power) then
  Qg_min_full=dmin1(Pe_rated/dmax1(omega,1.d-15),max_lss_torque)
  Qg_max_full=Qg_min_full
else
  Qg_min_full=Qg_rated
  Qg_max_full=Qg_rated
endif
! Limits for partial load that opens in both ends
ommin1=omega_ref_min
ommin2=omega_ref_min/rel_sp_open_Qg
ommax1=(2.d0*rel_sp_open_Qg-1.d0)*omega_ref_max
ommax2=rel_sp_open_Qg*omega_ref_max
x=switch_spline(omegafilt,ommin1,ommin2)
Qg_min_partial=dmin1(Kopt*omegafilt**2*x,Kopt*ommax1**2)
x=switch_spline(omegafilt,ommax1,ommax2)
Qg_max_partial=dmax1(Kopt*omegafilt**2*(1.d0-x)+Qg_max_full*x,Kopt*ommin2**2)
! Switch based on pitch
switch1=switch_spline(meanpitang,switch1_pitang_lower,switch1_pitang_upper)
switch1=lowpass1orderfilt(deltat,stepno,switchfirstordervar,switch1)
! Interpolation between partial and full load torque limits based on switch 1
PID_gen_var.outmin=(1.d0-switch1)*Qg_min_partial+switch1*Qg_min_full
PID_gen_var.outmax=(1.d0-switch1)*Qg_max_partial+switch1*Qg_max_full
if (PID_gen_var.outmin.gt.PID_gen_var.outmax) PID_gen_var.outmin=PID_gen_var.outmax
! Compute PID feedback to generator torque
kgain_torque=1.d0
Qgen_ref=PID(stepno,deltat,kgain_torque,PID_gen_var,omega_err_filt_speed)
!-------------------------------------------------------------------------------------------------
! Control of cut-in regarding generator torque
!-------------------------------------------------------------------------------------------------
if (t_cutin.gt.0.d0) then
  if (generator_cutin) then
    x=switch_spline(time,t_generator_cutin,t_generator_cutin+t_cutin_delay)
    Qgen_ref=Qgen_ref*x
  else
    Qgen_ref=0.d0
  endif
endif
!-------------------------------------------------------------------------------------------------
! Control of cut-out regarding generator torque
!-------------------------------------------------------------------------------------------------
if ((t_cutout.gt.0.d0).and.(time.gt.t_cutout)) then
  Qgen_ref=lowpass1orderfilt(deltat,stepno,torquefirstordervar,0.d0)
else
  dummy=lowpass1orderfilt(deltat,stepno,torquefirstordervar,Qgen_ref)
endif
!-------------------------------------------------------------------------------------------------
! Reference electrical power
!-------------------------------------------------------------------------------------------------
Pe_ref=Qgen_ref*omega
!-------------------------------------------------------------------------------------------------
! Active DT damping based on notch filtered of rotor speed
!-------------------------------------------------------------------------------------------------
if ((DT_damp_gain.gt.0.d0).and.(DT_damper_filt.f0.gt.0.d0)) then
  omega_dtfilt=bandpassfilt(deltat,stepno,DT_damper_filt,omega)
  if (t_cutin.gt.0.d0) then
    if (generator_cutin) then
      x=switch_spline(time,t_generator_cutin+t_cutin_delay,t_generator_cutin+2.d0*t_cutin_delay)
      Qdamp_ref=DT_damp_gain*omega_dtfilt*x
      Qgen_ref=dmin1(dmax1(Qgen_ref+Qdamp_ref,0.d0),max_lss_torque)
    endif
  else
    Qdamp_ref=DT_damp_gain*omega_dtfilt
    Qgen_ref=dmin1(dmax1(Qgen_ref+Qdamp_ref,0.d0),max_lss_torque)
  endif
endif
!**************************************************************************************************
! PID regulation of collective pitch angle
!**************************************************************************************************
! Reference speed is equal rated speed
omega_err_filt_pitch=omegafilt-omega_ref_max
! Limits
PID_pit_var.outmin=theta_min
PID_pit_var.outmax=pitch_stopang
! Aerodynamic gain scheduling
if (kk2.gt.0.d0) then
  aero_gain=1.d0+meanpitangfilt/kk1+meanpitangfilt**2/kk2
else
  aero_gain=1.d0+meanpitangfilt/kk1
endif
! Nonlinear gain to avoid large rotor speed excursion
kgain_pitch=(omega_err_filt_pitch**2/(omega_ref_max*(rel_limit-1.d0))**2+1.d0)/aero_gain
!-------------------------------------------------------------------------------------------------
! Control of cut-in regarding pitch
!-------------------------------------------------------------------------------------------------
if (t_cutin.gt.0.d0) then
  if (time.lt.t_cutin) then
    PID_pit_var.outmin=pitch_stopang
    PID_pit_var.outmax=pitch_stopang
    kgain_pitch=0.d0
    dummy=lowpass1orderfilt(deltat,stepno,cutinfirstordervar,omega-omega_ref_min)
  else
    if (.not.generator_cutin) then
      kgain_pitch(1:3,1)=0.25d0*kgain_pitch(1:3,1)
      kgain_pitch(1:3,2)=0.d0
      omega_err_filt_pitch=omegafilt-omega_ref_min
      x=lowpass1orderfilt(deltat,stepno,cutinfirstordervar,omega-omega_ref_min)
      if (dabs(x).lt.omega_ref_min*1.d-2) then
        generator_cutin=.true.
        t_generator_cutin=time
      endif
    else
      x=switch_spline(time,t_generator_cutin,t_generator_cutin+t_cutin_delay)
      omega_err_filt_pitch=omegafilt-(omega_ref_min*(1.d0-x)+omega_ref_max*x)
      kgain_pitch(1:3,1)=0.25d0*kgain_pitch(1:3,1)+kgain_pitch(1:3,1)*0.75d0*x
      kgain_pitch(1:3,2)=kgain_pitch(1:3,2)*x
    endif
  endif
endif
!-------------------------------------------------------------------------------------------------
! Control of cut-out regarding pitch
!-------------------------------------------------------------------------------------------------
if ((t_cutout.gt.0.d0).and.(time.gt.t_cutout+pitch_stopdelay)) then
  select case(pitch_stoptype)
    case(1) ! Normal 2-step stop situation
      PID_pit_var.outmax=pitch_stopang
      PID_pit_var.outmin=pitch_stopang
      if (time.gt.t_cutout+pitch_stopdelay+pitch_stopdelay2) then
        PID_pit_var.velmax=pitch_stopvelmax2
      else
        PID_pit_var.velmax=pitch_stopvelmax
      endif
    case(2) ! Exponential decay approach
      PID_pit_var.outmax=pitch_stopang
      PID_pit_var.outmin=pitch_stopang
      if ((time-(t_cutout+pitch_stopdelay))/pitch_stopdelay2.lt.10.d0) then
        PID_pit_var.velmax=pitch_stopang/pitch_stopdelay2&
                          *dexp(-(time-(t_cutout+pitch_stopdelay))/pitch_stopdelay2)
      else
        PID_pit_var.velmax=0.d0
      endif
      if (PID_pit_var.velmax.gt.pitch_stopvelmax) PID_pit_var.velmax=pitch_stopvelmax
      if (PID_pit_var.velmax.lt.pitch_stopvelmax2) PID_pit_var.velmax=pitch_stopvelmax2
    case default
      write(6,'(a,i2,a)') ' *** ERROR *** Stop type ',pitch_stoptype,' not known'
      stop
  end select
endif
!-------------------------------------------------------------------------------------------------
! Compute PID feedback to generator torque
!-------------------------------------------------------------------------------------------------
if (DT_mode_filt.f0.gt.0.d0) then
  e_pitch(1)=notch2orderfilt(deltat,stepno,DT_mode_filt,omega_err_filt_pitch)
  e_pitch(2)=notch2orderfilt(deltat,stepno,pwr_DT_mode_filt,Pe_ref-Pe_rated)
else
  e_pitch(1)=omega_err_filt_pitch
  e_pitch(2)=Pe_ref-Pe_rated
endif
theta_col_ref=PID2(stepno,deltat,kgain_pitch,PID_pit_var,e_pitch)
thetaref=theta_col_ref
!**************************************************************************************************
! Output
!**************************************************************************************************
array2( 1)=Qgen_ref             !    1: Generator torque reference            [Nm]
array2( 2)=thetaref(1)          !    2: Pitch angle reference of blade 1      [rad]
array2( 3)=thetaref(2)          !    3: Pitch angle reference of blade 2      [rad]
array2( 4)=thetaref(3)          !    4: Pitch angle reference of blade 3      [rad]
array2( 5)=Pe_ref               !    5: Power reference                       [W]
array2( 6)=WSPfilt              !    6: Filtered wind speed                   [m/s]
array2( 7)=omegafilt            !    7: Filtered rotor speed                  [rad/s]
array2( 8)=omega_err_filt_speed !    8: Filtered rotor speed error for torque [rad/s]
array2( 9)=omega_dtfilt         !    9: Bandpass filtered rotor speed         [rad/s]
array2(10)=PID_gen_var.outpro   !   10: Proportional term of torque contr.    [Nm]
array2(11)=PID_gen_var.outset   !   11: Integral term of torque controller    [Nm]
array2(12)=PID_gen_var.outmin   !   12: Minimum limit of torque               [Nm]
array2(13)=PID_gen_var.outmax   !   13: Maximum limit of torque               [Nm]
array2(14)=switch1              !   14: Torque limit switch based on pitch    [-]
array2(15)=omega_err_filt_pitch !   15: Filtered rotor speed error for pitch  [rad/s]
array2(16)=e_pitch(2)           !   16: Power error for pitch                 [W]
array2(17)=PID_pit_var.outpro   !   17: Proportional term of pitch controller [rad]
array2(18)=PID_pit_var.outset   !   18: Integral term of pitch controller     [rad]
array2(19)=PID_pit_var.outmin   !   19: Minimum limit of pitch                [rad]
array2(20)=PID_pit_var.outmax   !   20: Maximum limit of pitch                [rad]
array2(21)=Qdamp_ref            !   21: Torque reference from DT damper       [Nm]
return
end subroutine update_regulation
!**************************************************************************************************
\end{verbatim}
}

\pagebreak
\section{Functional routines in risoe\_controller\_fcns.f90}
{\scriptsize
\begin{verbatim}
module risoe_controller_fcns
! Constants
real*8 pi,degrad,raddeg
parameter(pi=3.14159265358979,degrad=0.0174532925,raddeg=57.2957795131)
integer*4 maxwplines
parameter(maxwplines=100)
! Types
type Tfirstordervar
  real*8 tau,x1,x1_old,y1,y1_old
  integer*4 stepno1
end type Tfirstordervar
type Tlowpass2order
  real*8 zeta,f0,x1,x2,x1_old,x2_old,y1,y2,y1_old,y2_old
  integer*4 stepno1
end type Tlowpass2order
type Tnotch2order
  real*8::zeta1=0.1
  real*8::zeta2=0.001
  real*8 f0,x1,x2,x1_old,x2_old,y1,y2,y1_old,y2_old
  integer*4 stepno1
end type Tnotch2order
type Tbandpassfilt
  real*8::zeta=0.02
  real*8::tau=0.0
  real*8 f0,x1,x2,x1_old,x2_old,y1,y2,y1_old,y2_old
  integer*4 stepno1
end type Tbandpassfilt
type Tpidvar
  real*8 Kpro,Kdif,Kint,outmin,outmax,velmax,error1,outset1,outres1
  integer*4 stepno1
  real*8 outset,outpro,outdif,error1_old,outset1_old,outres1_old,outres
end type Tpidvar
type Tpid2var
  real*8 Kpro(2),Kdif(2),Kint(2),outmin,outmax,velmax,error1(2),outset1,outres1
  integer*4 stepno1
  real*8 outset,outpro,outdif,error1_old(2),outset1_old,outres1_old,outres
end type Tpid2var
type Twpdata
  real*8 wpdata(maxwplines,2)
  integer*4 lines
end type Twpdata
! Variables
integer*4 stepno
logical const_power
real*8 deltat,time_old
real*8 omega_ref_max,omega_ref_min,Pe_rated,Qg_rated,pitch_stopang,max_lss_torque
real*8 Kopt,rel_sp_open_Qg
real*8 kk1,kk2,rel_limit
real*8 switch1_pitang_lower,switch1_pitang_upper,switch1
real*8 DT_damp_gain
logical::generator_cutin=.false.
real*8 t_cutin,t_generator_cutin,t_cutin_delay
integer*4 pitch_stoptype
real*8 t_cutout,pitch_stopdelay,pitch_stopdelay2,pitch_stopvelmax,pitch_stopvelmax2
type(Tlowpass2order) omega2ordervar
type(Tnotch2order) DT_mode_filt
type(Tnotch2order) pwr_DT_mode_filt
type(Tbandpassfilt) DT_damper_filt
type(Tpid2var) PID_pit_var
type(Tpidvar) PID_gen_var
type(Twpdata) OPdatavar
type(Tfirstordervar) wspfirstordervar
type(Tfirstordervar) pitchfirstordervar
type(Tfirstordervar) torquefirstordervar
type(Tfirstordervar) switchfirstordervar
type(Tfirstordervar) cutinfirstordervar
!**************************************************************************************************
contains
!**************************************************************************************************
function switch_spline(x,x0,x1)
implicit none
! A function that goes from 0 at x0 to 1 at x1
real*8 switch_spline,x,x0,x1
if (x0.ge.x1) then
  if (x.lt.x0) then
    switch_spline=0.d0
  else
    switch_spline=1.d0
  endif
elseif (x0.gt.x1) then
  switch_spline=0.d0
else
  if (x.lt.x0) then
    switch_spline=0.d0
  elseif (x.gt.x1) then
    switch_spline=1.d0
  else
    switch_spline=2.d0/(-x1+x0)**3*x**3+(-3.d0*x0-3.d0*x1)/(-x1+x0)**3*x**2&
                 +6.d0*x1*x0/(-x1+x0)**3*x+(x0-3.d0*x1)*x0**2/(-x1+x0)**3
  endif
endif
return
end function switch_spline
!**************************************************************************************************
function interpolate(x,x0,x1,f0,f1)
implicit none
real*8 interpolate,x,x0,x1,f0,f1
if (x0.eq.x1) then
  interpolate=f0
else
  interpolate=(x-x1)/(x0-x1)*f0+(x-x0)/(x1-x0)*f1
endif
return
end function interpolate
!**************************************************************************************************
function GetOptiPitch(wsp)
implicit none
real*8 GetOptiPitch,wsp
! local vars
real*8 x,x0,x1,f0,f1,pitch
integer*4 i
i=1
do while((OPdatavar.wpdata(i,1).le.wsp).and.(i.le.OPdatavar.lines))
  i=i+1
enddo
if (i.eq.1) then
  GetOptiPitch=OPdatavar.wpdata(1,2)
elseif (i.gt.OPdatavar.lines) then
  GetOptiPitch=OPdatavar.wpdata(OPdatavar.lines,2)
else
  x=wsp
  x0=OPdatavar.wpdata(i-1,1)
  x1=OPdatavar.wpdata(i,1)
  f0=OPdatavar.wpdata(i-1,2)
  f1=OPdatavar.wpdata(i,2)
  Pitch=interpolate(x,x0,x1,f0,f1)
  GetOptiPitch=Pitch
endif
return
end function GetOptiPitch
!**************************************************************************************************
function lowpass1orderfilt(dt,stepno,filt,x)
implicit none
integer*4 stepno
real*8 lowpass1orderfilt,dt,x,y,a1,b1,b0,tau
type(Tfirstordervar) filt
! Step
if ((stepno.eq.1).and.(stepno.gt.filt.stepno1)) then
  filt.x1_old=x
  filt.y1_old=x
  y=x
else
  if (stepno.gt.filt.stepno1) then
    filt.x1_old=filt.x1
    filt.y1_old=filt.y1
  endif
  tau=filt.tau
  a1 = (2 * tau - dt) / (2 * tau + dt)
  b0 = dt / (2 * tau + dt)
  b1 = b0
  y=a1*filt.y1_old+b0*x+b1*filt.x1_old
endif
! Save previous values
filt.x1=x
filt.y1=y
filt.stepno1=stepno
! Output
lowpass1orderfilt=y
return
end function lowpass1orderfilt
!**************************************************************************************************
function lowpass2orderfilt(dt,stepno,filt,x)
implicit none
real*8 lowpass2orderfilt(2),dt,x
integer*4 stepno
type(Tlowpass2order) filt
! local vars
real*8 y,f0,zeta,a1,a2,b0,b1,b2,denom
! Step
if ((stepno.eq.1).and.(stepno.gt.filt.stepno1)) then
  filt.x1=x
  filt.x2=x
  filt.x1_old=filt.x1
  filt.x2_old=filt.x2
  filt.y1=x
  filt.y2=x
  filt.y1_old=filt.y1
  filt.y2_old=filt.y2
  y=x
else
  if (stepno.gt.filt.stepno1) then
    filt.x1_old=filt.x1
    filt.x2_old=filt.x2
    filt.y1_old=filt.y1
    filt.y2_old=filt.y2
  endif
  f0=filt.f0
  zeta=filt.zeta
  denom=3.d0+6.d0*zeta*pi*f0*dt+4.d0*pi**2*f0**2*dt**2
  a1=(6.d0-4.d0*pi**2*f0**2*dt**2)/denom
  a2=(-3.d0+6.d0*zeta*pi*f0*dt-4.d0*pi**2*f0**2*dt**2)/denom
  b0=4.d0*pi**2*f0**2*dt**2/denom
  b1=b0
  b2=b0
  y=a1*filt.y1_old+a2*filt.y2_old+b0*x+b1*filt.x1_old+b2*filt.x2_old
endif
! Save previous values
filt.x2=filt.x1
filt.x1=x
filt.y2=filt.y1
filt.y1=y
filt.stepno1=stepno
! Output
lowpass2orderfilt(1)=y
lowpass2orderfilt(2)=0.5d0*(y-filt.y2_old)/dt
return
end function lowpass2orderfilt
!**************************************************************************************************
function notch2orderfilt(dt,stepno,filt,x)
implicit none
real*8 notch2orderfilt,dt,x
integer*4 stepno
type(Tnotch2order) filt
! local vars
real*8 y,f0,zeta1,zeta2,a1,a2,b0,b1,b2,denom
! Step
if ((stepno.eq.1).and.(stepno.gt.filt.stepno1)) then
  filt.x1=x
  filt.x2=x
  filt.x1_old=filt.x1
  filt.x2_old=filt.x2
  filt.y1=x
  filt.y2=x
  filt.y1_old=filt.y1
  filt.y2_old=filt.y2
  y=x
else
  if (stepno.gt.filt.stepno1) then
    filt.x1_old=filt.x1
    filt.x2_old=filt.x2
    filt.y1_old=filt.y1
    filt.y2_old=filt.y2
  endif
  f0=filt.f0
  zeta1=filt.zeta1
  zeta2=filt.zeta2
  denom=3.d0+6.d0*zeta1*pi*f0*dt+4.d0*pi**2*f0**2*dt**2
  a1=(6.d0-4.d0*pi**2*f0**2*dt**2)/denom
  a2=(-3.d0+6.d0*zeta1*pi*f0*dt-4.d0*pi**2*f0**2*dt**2)/denom
  b0=(3.d0+6.d0*zeta2*pi*f0*dt+4.d0*pi**2*f0**2*dt**2)/denom
  b1=(-6.d0+4.d0*pi**2*f0**2*dt**2)/denom
  b2=(3.d0-6.d0*zeta2*pi*f0*dt+4.d0*pi**2*f0**2*dt**2)/denom
  y=a1*filt.y1_old+a2*filt.y2_old+b0*x+b1*filt.x1_old+b2*filt.x2_old
endif
! Save previous values
filt.x2=filt.x1
filt.x1=x
filt.y2=filt.y1
filt.y1=y
filt.stepno1=stepno
! Output
notch2orderfilt=y
return
end function notch2orderfilt
!**************************************************************************************************
function bandpassfilt(dt,stepno,filt,x)
implicit none
real*8 bandpassfilt,dt,x
integer*4 stepno
type(Tbandpassfilt) filt
! local vars
real*8 y,f0,zeta,tau,a1,a2,b0,b1,b2,denom
! Step
if ((stepno.eq.1).and.(stepno.gt.filt.stepno1)) then
  filt.x1=x
  filt.x2=x
  filt.x1_old=filt.x1
  filt.x2_old=filt.x2
  filt.y1=x
  filt.y2=x
  filt.y1_old=filt.y1
  filt.y2_old=filt.y2
  y=x
else
  if (stepno.gt.filt.stepno1) then
    filt.x1_old=filt.x1
    filt.x2_old=filt.x2
    filt.y1_old=filt.y1
    filt.y2_old=filt.y2
  endif
  f0=filt.f0
  zeta=filt.zeta
  tau=filt.tau
  denom=3.d0+6.d0*zeta*pi*f0*dt+4.d0*pi**2*f0**2*dt**2
  a1=-(-6.d0+4.d0*pi**2*f0**2*dt**2)/denom
  a2=-(3.d0-6.d0*zeta*pi*f0*dt+4.d0*pi**2*f0**2*dt**2)/denom
  b0=-(-6.d0*zeta*pi*f0*dt-12.d0*zeta*pi*f0*tau)/denom
  b1=-24.d0*zeta*pi*f0*tau/denom
  b2=-(6.d0*zeta*pi*f0*dt-12.d0*zeta*pi*f0*tau)/denom
  y=a1*filt.y1_old+a2*filt.y2_old+b0*x+b1*filt.x1_old+b2*filt.x2_old
endif
! Save previous values
filt.x2=filt.x1
filt.x1=x
filt.y2=filt.y1
filt.y1=y
filt.stepno1=stepno
! Output
bandpassfilt=y
return
end function bandpassfilt
!**************************************************************************************************
function PID(stepno,dt,kgain,PIDvar,error)
implicit none
integer*4 stepno
real*8 PID,dt,kgain(3),error
type(Tpidvar) PIDvar
! Local vars
real*8 eps
parameter(eps=1.d-6)
! Initiate
if (stepno.eq.1) then
  PIDvar.outset1=0
  PIDvar.outres1=0
  PIDvar.error1=0
  PIDvar.error1_old=0.0
  PIDvar.outset1_old=0.0
  PIDvar.outres1_old=0.0
endif
! Save previous values
if (stepno.gt.PIDvar.stepno1) then
  PIDvar.outset1_old=PIDvar.outset1
  PIDvar.outres1_old=PIDvar.outres1
  PIDvar.error1_old=PIDvar.error1
endif
! Update the integral term
PIDvar.outset=PIDvar.outset1_old+0.5d0*(error+PIDvar.error1)*Kgain(2)*PIDvar.Kint*dt
! Update proportional term
PIDvar.outpro=Kgain(1)*PIDvar.Kpro*0.5d0*(error+PIDvar.error1)
! Update differential term
PIDvar.outdif=Kgain(3)*PIDvar.Kdif*(error-PIDvar.error1_old)/dt
! Sum to up
PIDvar.outres=PIDvar.outset+PIDvar.outpro+PIDvar.outdif
! Satisfy hard limits
if (PIDvar.outres.lt.PIDvar.outmin) then
  PIDvar.outres=PIDvar.outmin
elseif (PIDvar.outres.gt.PIDvar.outmax) then
  PIDvar.outres=PIDvar.outmax
endif
! Satisfy max velocity
if (PIDvar.velmax.gt.eps) then
    if ((abs(PIDvar.outres-PIDvar.outres1_old)/dt).gt.PIDvar.velmax) &
      PIDvar.outres=PIDvar.outres1_old+dsign(PIDvar.velmax*dt,PIDvar.outres-PIDvar.outres1_old)
endif
! Anti-windup on integral term and save results
PIDvar.outset1=PIDvar.outres-PIDvar.outpro-PIDvar.outdif
PIDvar.outres1=PIDvar.outres
PIDvar.error1=error
PIDvar.stepno1=stepno
! Set output
if (stepno.eq.0) then
  PID=0
else
  PID=PIDvar.outres
endif
return
end function PID
!**************************************************************************************************
function PID2(stepno,dt,kgain,PIDvar,error)
implicit none
integer*4 stepno
real*8 PID2,dt,kgain(3,2),error(2)
type(Tpid2var) PIDvar
! Local vars
real*8 eps
parameter(eps=1.d-6)
! Initiate
if (stepno.eq.1) then
  PIDvar.outset1=0
  PIDvar.outres1=0
  PIDvar.error1=0
  PIDvar.error1_old=0.0
  PIDvar.outset1_old=0.0
  PIDvar.outres1_old=0.0
endif
! Save previous values
if (stepno.gt.PIDvar.stepno1) then
  PIDvar.outset1_old=PIDvar.outset1
  PIDvar.outres1_old=PIDvar.outres1
  PIDvar.error1_old=PIDvar.error1
endif
! Update the integral term
PIDvar.outset=PIDvar.outset1_old+0.5d0*dt*(Kgain(2,1)*PIDvar.Kint(1)*(error(1)+PIDvar.error1(1))&
                                          +Kgain(2,2)*PIDvar.Kint(2)*(error(2)+PIDvar.error1(2)))
! Update proportional term
PIDvar.outpro=0.5d0*(Kgain(1,1)*PIDvar.Kpro(1)*(error(1)+PIDvar.error1(1))&
                    +Kgain(1,2)*PIDvar.Kpro(2)*(error(2)+PIDvar.error1(2)))
! Update differential term
PIDvar.outdif=(Kgain(3,1)*PIDvar.Kdif(1)*(error(1)-PIDvar.error1_old(1)))/dt
! Sum to up
PIDvar.outres=PIDvar.outset+PIDvar.outpro+PIDvar.outdif
! Satisfy hard limits
if (PIDvar.outres.lt.PIDvar.outmin) then
  PIDvar.outres=PIDvar.outmin
elseif (PIDvar.outres.gt.PIDvar.outmax) then
  PIDvar.outres=PIDvar.outmax
endif
! Satisfy max velocity
if (PIDvar.velmax.gt.eps) then
    if ((abs(PIDvar.outres-PIDvar.outres1_old)/dt).gt.PIDvar.velmax) &
      PIDvar.outres=PIDvar.outres1_old+dsign(PIDvar.velmax*dt,PIDvar.outres-PIDvar.outres1_old)
endif
! Anti-windup on integral term and save results
PIDvar.outset1=PIDvar.outres-PIDvar.outpro-PIDvar.outdif
PIDvar.outres1=PIDvar.outres
PIDvar.error1=error
PIDvar.stepno1=stepno
! Set output
if (stepno.eq.0) then
  PID2=0
else
  PID2=PIDvar.outres
endif
return
end function PID2
!**************************************************************************************************
end module risoe_controller_fcns
\end{verbatim}
}

\pagebreak
\chapter{Discrete filters} \label{ch:filters}

This appendix contains the derivations of the discrete filters and a test of their validity. Some of the parameters of the filters are hardcoded in the current implementation as shown in the previous appendix. The values of these hardcoded parameters are repeated herein.

\section{First order filter}

In continuous form the first order low-pass filter can be written as
\begin{equation}\label{e:f1cont}
\dot{\bar x} + \tau \bar x = \tau x
\end{equation}
where $x=x(t)$ is the original signal, $\bar x(t)$ is the filtered signal, $\tau$ is the user-defined time constant of the filter, and $\dot{(\,)}=\d/dt$ denotes the time derivative. The discrete first order low-pass filter used in the controller is derived from this continuous formulation by approximating the states and their time derivatives as an average based on the previous and current step:
\begin{equation}\label{e:1storder}
x(t)\approx \frac{x_k+x_{k-1}}2 \;\;\;\;\mbox{and} \;\;\;\;\dot x(t)\approx \frac{x_k-x_{k-1}}{\Delta t}
\end{equation}
where $k$ is the index of the current time step, and $\Delta t$ is the time step length. Substitution into \eqref{e:f1cont} and rearranging the terms, the $f_1$-function is obtained as
\begin{equation}
\label{e:f1}
f_1 \left(\tau; \bar x_{k-1} , x_{k} , x_{k-1} \right) = a_1 \bar x_{k-1} + b_0 x_k + b_1 x_{k-1}
\end{equation}
where
\begin{equation}
\label{e:f1coef}
a_1=\frac {2\tau-\Delta t}{2\tau+\Delta t} , \;\;\;\; b_0 = \frac {\Delta t}{2\tau+\Delta t}, \;\;\;\; b_1 = b_0
\end{equation}

To test this discrete filter, the Finite-Difference Method for constructing periodic solutions \cite{Nayfeh95} has been used to obtain the period solutions to harmonic excitation of the filter with a large number of different excitation frequencies. Figure~\ref{f:f1} shows the amplitude and phase of these solutions (red circles) compared to the transfer function obtained by transformation of the continuous formulation \eqref{e:f1cont} into the frequency domain.

\begin{figure}[t]
\centerline{\epsfig{figure=f1.eps,width=0.95\textwidth} }
\caption{Test of the first order low-pass filter function \eqref{e:f1} with a time constant of $\tau=1$~s and a time step of $\Delta t = 0.125$~s. \label{f:f1}}
\end{figure}


\section{Second order filters}

The discrete second filters used in the controller are derived from the continuous formulations by approximating the states and their time derivatives as an average based on the two previous and current step:
\begin{align}\label{e:2ndorder}
x(t) \approx \frac{x_k+x_{k-1}+x_{k-2}}3,  \;\;\;\;\dot x(t)\approx \frac{x_k-x_{k-2}}{2\Delta t} \;\;\;\;\mbox{and} \;\;\;\;\ddot x(t)\approx \frac{x_k-2 x_{k-1}+ x_{k-2}}{\Delta t^2}
\end{align}
The continuous form of the second order low-pass filter is
\begin{equation}
\label{e:f2cont}
\ddot{\bar x} + 2 \zeta \omega \dot{\bar x} + \omega^2\bar x = \omega^2 x
\end{equation}
where again $x=x(t)$ and $\bar x(t)$ are the original and filtered signals, respectively. The parameters $\zeta$ and $\omega$ are the user-defined damping ratio and frequency of the filter. Substitution of \eqref{e:2ndorder} into this equations yields the discrete second order low-pass filter function:
\begin{equation}
\label{e:f2}
f_2 \left(\zeta, \omega; \bar x_{k-1} , \bar x_{k-2} , x_{k} , x_{k-1} , x_{k-2} \right) = a_1 \bar x_{k-1} + a_2 \bar x_{k-2} + b_0 x_k + b_1 x_{k-1}+ b_2 x_{k-2}
\end{equation}
where
\begin{gather}\nonumber
a_1=\frac {6-\omega^2 \Delta t^2}{d},
\;\;\;\;
a_2=\frac {-3+3\,\zeta\,\omega\,\Delta t-\omega^2\Delta t^2}{d}, \\
\label{e:f2coef}
b_0= \frac {\omega^2\Delta t^2}{d}, \;\;\;\; b_1 = b_0\;\;\;\; b_2 = b_0
\end{gather}
where the common denominator is $d=3+3 \zeta \omega \Delta t+\omega^2 \Delta t^2$.

The continuous form of the second order notch filter is
\begin{equation}
\label{e:fncont}
\ddot{\bar x} + 2 \zeta_1 \omega \dot{\bar x} + \omega^2\bar x = \ddot{x} + 2 \zeta_2 \omega \dot{x} + \omega^2 x
\end{equation}
where $\zeta_1=0.1$, $\zeta_2=0.001$ and $\omega$ are the hardcoded damping ratios and the user-defined frequency of the notch filter, respectively. Substitution of \eqref{e:2ndorder} into this equations yields the discrete second order notch filter function:
\begin{equation}
\label{e:fn}
f_n \left(\zeta_{1}, \zeta_{2}, \omega; \bar x_{k-1} , \bar x_{k-2} , x_{k} , x_{k-1} , x_{k-2} \right) = a_1 \bar x_{k-1} + a_2 \bar x_{k-2} + b_0 x_k + b_1 x_{k-1}+ b_2 x_{k-2}
\end{equation}
where
\begin{gather}\nonumber
a_1=-{\frac {-6+{\omega}^{2}{\Delta t}^{2}}{d}},
\;\;\;\;
a_2=-{\frac {3-3\,\zeta_{1}\,\omega\,\Delta t+{\omega}^{2}{\Delta t}^{2}}{d}}, \\
\label{e:fncoef}
b_0 = {\frac {3+3\,\zeta_{2}\,\omega\,\Delta t+{\omega}^{2}{\Delta t}^{2}}{d}}, \;\;\;\;
b_1 = -a_1,\;\;\;\;
b_2 = {\frac {3-3\,\zeta_{2}\,\omega\,\Delta t+{\omega}^{2}{\Delta t}^{2}}{d}}
\end{gather}
where the common denominator is $d=3+3\,\zeta_{1}\,\omega\,\Delta t+{\omega}^{2}{\Delta t}^{2}$.

The continuous form of the second order band-pass filter used in the controller is
\begin{equation}
\label{e:fpcont}
\ddot{\bar x} + 2 \zeta \omega \dot{\bar x} + \omega^2\bar x = 2 \zeta \omega \left(\dot{x} + \tau \ddot{x}\right)
\end{equation}
where $\zeta=0.02$ and $\omega$ are the hardcoded damping ratio and the user-defined frequency of the band-pass filter, respectively. The additional parameter $\tau$ is a time constant which in the current implementation is hardcoded to zero. Substitution of \eqref{e:2ndorder} into this equations yields the discrete second order band-pass filter function:
\begin{equation}
\label{e:fp}
f_p \left(\zeta, \omega; \bar x_{k-1} , \bar x_{k-2} , x_{k} , x_{k-1} , x_{k-2} \right) = a_1 \bar x_{k-1} + a_2 \bar x_{k-2} + b_0 x_k + b_1 x_{k-1}+ b_2 x_{k-2}
\end{equation}
where
\begin{gather}\nonumber
a_1 =-{\frac {-6+{\omega}^{2}{\Delta t}^{2}}{d}},\;\;\;\;
a_2 =-{\frac {3-3\,\zeta\,\omega\,\Delta t+{\omega}^{2}{\Delta t}^{2}}{d}}, \\
\label{e:fpcoef}
b_0 =3\,{\frac {\zeta\,\omega\, \left( \Delta t+2\,\tau \right) }{d}},\;\;\;\;
b_1 =-12\,{\frac {\zeta\,\omega\,\tau}{d}},\;\;\;\;
b_2 =-3\,{\frac {\zeta\,\omega\, \left( \Delta t-2\,\tau \right) }{d}}
\end{gather}
where the common denominator is $d=3+3 \zeta \omega \Delta t+\omega^2 \Delta t^2$. Notice that the hardcoded parameter $\tau$ is not included in the list of parameters in the function call.

Test and validation of these second order filters are shown in Figures~\ref{f:f2}~--~\ref{f:fp}.

\begin{figure}[b!]
\centerline{\epsfig{figure=f2.eps,width=0.95\textwidth} }
\caption{Test of the second order low-pass filter function \eqref{e:f2} with a damping ratio of $\zeta=0.7$, a frequency of $\omega=0.15$~Hz and a time step of $\Delta t = 0.125$~s. \label{f:f2}}
\end{figure}

\begin{figure}[t]
\centerline{\epsfig{figure=fn.eps,width=0.95\textwidth} }
\caption{Test of the second order notch filter function \eqref{e:fn} with a frequency of $\omega=0.625$~Hz and a time step of $\Delta t = 0.04$~s. \label{f:fn}}
\end{figure}

\begin{figure}[t]
\centerline{\epsfig{figure=fp.eps,width=0.95\textwidth} }
\caption{Test of the second order band-pass filter function \eqref{e:fp} with a frequency of $\omega=0.625$~Hz and a time step of $\Delta t = 0.04$~s. \label{f:fp}}
\end{figure}
